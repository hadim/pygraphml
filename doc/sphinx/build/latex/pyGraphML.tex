% Generated by Sphinx.
\def\sphinxdocclass{report}
\documentclass[letterpaper,10pt,english]{sphinxmanual}
\usepackage[utf8]{inputenc}
\DeclareUnicodeCharacter{00A0}{\nobreakspace}
\usepackage[T1]{fontenc}
\usepackage{babel}
\usepackage{times}
\usepackage[Bjarne]{fncychap}
\usepackage{longtable}
\usepackage{sphinx}


\title{pyGraphML Documentation}
\date{July 26, 2011}
\release{0.1}
\author{Hadrien Mary}
\newcommand{\sphinxlogo}{}
\renewcommand{\releasename}{Release}
\makeindex

\makeatletter
\def\PYG@reset{\let\PYG@it=\relax \let\PYG@bf=\relax%
    \let\PYG@ul=\relax \let\PYG@tc=\relax%
    \let\PYG@bc=\relax \let\PYG@ff=\relax}
\def\PYG@tok#1{\csname PYG@tok@#1\endcsname}
\def\PYG@toks#1+{\ifx\relax#1\empty\else%
    \PYG@tok{#1}\expandafter\PYG@toks\fi}
\def\PYG@do#1{\PYG@bc{\PYG@tc{\PYG@ul{%
    \PYG@it{\PYG@bf{\PYG@ff{#1}}}}}}}
\def\PYG#1#2{\PYG@reset\PYG@toks#1+\relax+\PYG@do{#2}}

\def\PYG@tok@gd{\def\PYG@tc##1{\textcolor[rgb]{0.63,0.00,0.00}{##1}}}
\def\PYG@tok@gu{\let\PYG@bf=\textbf\def\PYG@tc##1{\textcolor[rgb]{0.50,0.00,0.50}{##1}}}
\def\PYG@tok@gt{\def\PYG@tc##1{\textcolor[rgb]{0.00,0.25,0.82}{##1}}}
\def\PYG@tok@gs{\let\PYG@bf=\textbf}
\def\PYG@tok@gr{\def\PYG@tc##1{\textcolor[rgb]{1.00,0.00,0.00}{##1}}}
\def\PYG@tok@cm{\let\PYG@it=\textit\def\PYG@tc##1{\textcolor[rgb]{0.25,0.50,0.56}{##1}}}
\def\PYG@tok@vg{\def\PYG@tc##1{\textcolor[rgb]{0.73,0.38,0.84}{##1}}}
\def\PYG@tok@m{\def\PYG@tc##1{\textcolor[rgb]{0.13,0.50,0.31}{##1}}}
\def\PYG@tok@mh{\def\PYG@tc##1{\textcolor[rgb]{0.13,0.50,0.31}{##1}}}
\def\PYG@tok@cs{\def\PYG@tc##1{\textcolor[rgb]{0.25,0.50,0.56}{##1}}\def\PYG@bc##1{\colorbox[rgb]{1.00,0.94,0.94}{##1}}}
\def\PYG@tok@ge{\let\PYG@it=\textit}
\def\PYG@tok@vc{\def\PYG@tc##1{\textcolor[rgb]{0.73,0.38,0.84}{##1}}}
\def\PYG@tok@il{\def\PYG@tc##1{\textcolor[rgb]{0.13,0.50,0.31}{##1}}}
\def\PYG@tok@go{\def\PYG@tc##1{\textcolor[rgb]{0.19,0.19,0.19}{##1}}}
\def\PYG@tok@cp{\def\PYG@tc##1{\textcolor[rgb]{0.00,0.44,0.13}{##1}}}
\def\PYG@tok@gi{\def\PYG@tc##1{\textcolor[rgb]{0.00,0.63,0.00}{##1}}}
\def\PYG@tok@gh{\let\PYG@bf=\textbf\def\PYG@tc##1{\textcolor[rgb]{0.00,0.00,0.50}{##1}}}
\def\PYG@tok@ni{\let\PYG@bf=\textbf\def\PYG@tc##1{\textcolor[rgb]{0.84,0.33,0.22}{##1}}}
\def\PYG@tok@nl{\let\PYG@bf=\textbf\def\PYG@tc##1{\textcolor[rgb]{0.00,0.13,0.44}{##1}}}
\def\PYG@tok@nn{\let\PYG@bf=\textbf\def\PYG@tc##1{\textcolor[rgb]{0.05,0.52,0.71}{##1}}}
\def\PYG@tok@no{\def\PYG@tc##1{\textcolor[rgb]{0.38,0.68,0.84}{##1}}}
\def\PYG@tok@na{\def\PYG@tc##1{\textcolor[rgb]{0.25,0.44,0.63}{##1}}}
\def\PYG@tok@nb{\def\PYG@tc##1{\textcolor[rgb]{0.00,0.44,0.13}{##1}}}
\def\PYG@tok@nc{\let\PYG@bf=\textbf\def\PYG@tc##1{\textcolor[rgb]{0.05,0.52,0.71}{##1}}}
\def\PYG@tok@nd{\let\PYG@bf=\textbf\def\PYG@tc##1{\textcolor[rgb]{0.33,0.33,0.33}{##1}}}
\def\PYG@tok@ne{\def\PYG@tc##1{\textcolor[rgb]{0.00,0.44,0.13}{##1}}}
\def\PYG@tok@nf{\def\PYG@tc##1{\textcolor[rgb]{0.02,0.16,0.49}{##1}}}
\def\PYG@tok@si{\let\PYG@it=\textit\def\PYG@tc##1{\textcolor[rgb]{0.44,0.63,0.82}{##1}}}
\def\PYG@tok@s2{\def\PYG@tc##1{\textcolor[rgb]{0.25,0.44,0.63}{##1}}}
\def\PYG@tok@vi{\def\PYG@tc##1{\textcolor[rgb]{0.73,0.38,0.84}{##1}}}
\def\PYG@tok@nt{\let\PYG@bf=\textbf\def\PYG@tc##1{\textcolor[rgb]{0.02,0.16,0.45}{##1}}}
\def\PYG@tok@nv{\def\PYG@tc##1{\textcolor[rgb]{0.73,0.38,0.84}{##1}}}
\def\PYG@tok@s1{\def\PYG@tc##1{\textcolor[rgb]{0.25,0.44,0.63}{##1}}}
\def\PYG@tok@gp{\let\PYG@bf=\textbf\def\PYG@tc##1{\textcolor[rgb]{0.78,0.36,0.04}{##1}}}
\def\PYG@tok@sh{\def\PYG@tc##1{\textcolor[rgb]{0.25,0.44,0.63}{##1}}}
\def\PYG@tok@ow{\let\PYG@bf=\textbf\def\PYG@tc##1{\textcolor[rgb]{0.00,0.44,0.13}{##1}}}
\def\PYG@tok@sx{\def\PYG@tc##1{\textcolor[rgb]{0.78,0.36,0.04}{##1}}}
\def\PYG@tok@bp{\def\PYG@tc##1{\textcolor[rgb]{0.00,0.44,0.13}{##1}}}
\def\PYG@tok@c1{\let\PYG@it=\textit\def\PYG@tc##1{\textcolor[rgb]{0.25,0.50,0.56}{##1}}}
\def\PYG@tok@kc{\let\PYG@bf=\textbf\def\PYG@tc##1{\textcolor[rgb]{0.00,0.44,0.13}{##1}}}
\def\PYG@tok@c{\let\PYG@it=\textit\def\PYG@tc##1{\textcolor[rgb]{0.25,0.50,0.56}{##1}}}
\def\PYG@tok@mf{\def\PYG@tc##1{\textcolor[rgb]{0.13,0.50,0.31}{##1}}}
\def\PYG@tok@err{\def\PYG@bc##1{\fcolorbox[rgb]{1.00,0.00,0.00}{1,1,1}{##1}}}
\def\PYG@tok@kd{\let\PYG@bf=\textbf\def\PYG@tc##1{\textcolor[rgb]{0.00,0.44,0.13}{##1}}}
\def\PYG@tok@ss{\def\PYG@tc##1{\textcolor[rgb]{0.32,0.47,0.09}{##1}}}
\def\PYG@tok@sr{\def\PYG@tc##1{\textcolor[rgb]{0.14,0.33,0.53}{##1}}}
\def\PYG@tok@mo{\def\PYG@tc##1{\textcolor[rgb]{0.13,0.50,0.31}{##1}}}
\def\PYG@tok@mi{\def\PYG@tc##1{\textcolor[rgb]{0.13,0.50,0.31}{##1}}}
\def\PYG@tok@kn{\let\PYG@bf=\textbf\def\PYG@tc##1{\textcolor[rgb]{0.00,0.44,0.13}{##1}}}
\def\PYG@tok@o{\def\PYG@tc##1{\textcolor[rgb]{0.40,0.40,0.40}{##1}}}
\def\PYG@tok@kr{\let\PYG@bf=\textbf\def\PYG@tc##1{\textcolor[rgb]{0.00,0.44,0.13}{##1}}}
\def\PYG@tok@s{\def\PYG@tc##1{\textcolor[rgb]{0.25,0.44,0.63}{##1}}}
\def\PYG@tok@kp{\def\PYG@tc##1{\textcolor[rgb]{0.00,0.44,0.13}{##1}}}
\def\PYG@tok@w{\def\PYG@tc##1{\textcolor[rgb]{0.73,0.73,0.73}{##1}}}
\def\PYG@tok@kt{\def\PYG@tc##1{\textcolor[rgb]{0.56,0.13,0.00}{##1}}}
\def\PYG@tok@sc{\def\PYG@tc##1{\textcolor[rgb]{0.25,0.44,0.63}{##1}}}
\def\PYG@tok@sb{\def\PYG@tc##1{\textcolor[rgb]{0.25,0.44,0.63}{##1}}}
\def\PYG@tok@k{\let\PYG@bf=\textbf\def\PYG@tc##1{\textcolor[rgb]{0.00,0.44,0.13}{##1}}}
\def\PYG@tok@se{\let\PYG@bf=\textbf\def\PYG@tc##1{\textcolor[rgb]{0.25,0.44,0.63}{##1}}}
\def\PYG@tok@sd{\let\PYG@it=\textit\def\PYG@tc##1{\textcolor[rgb]{0.25,0.44,0.63}{##1}}}

\def\PYGZbs{\char`\\}
\def\PYGZus{\char`\_}
\def\PYGZob{\char`\{}
\def\PYGZcb{\char`\}}
\def\PYGZca{\char`\^}
\def\PYGZsh{\char`\#}
\def\PYGZpc{\char`\%}
\def\PYGZdl{\char`\$}
\def\PYGZti{\char`\~}
% for compatibility with earlier versions
\def\PYGZat{@}
\def\PYGZlb{[}
\def\PYGZrb{]}
\makeatother

\begin{document}

\maketitle
\tableofcontents
\phantomsection\label{index::doc}


pyGraphML is a GraphML parser written in Python. GraphML is a
comprehensive and easy-to-use file format for graphs.


\chapter{Contents}
\label{index:welcome-to-pygraphml-documentation}\label{index:contents}

\section{Getting started}
\label{gettingstarted:getting-started}\label{gettingstarted::doc}

\subsection{Using the bindings from the Python Interpreter}
\label{gettingstarted:using-the-bindings-from-the-python-interpreter}
The Tulip Python bindings can also be used through the classical Python Interpreter. But some setup has to be done
before importing the \code{tulip} module.

First, the path to the \code{tulip} module must be provided to Python.
In the following, \textless{}tulip\_install\_dir\textgreater{} represents the root directory of a Tulip installation.
The Tulip Python module is installed in the following directory according to your system :
\begin{itemize}
\item {} 
Linux : \textless{}tulip\_install\_dir\textgreater{}/lib

\item {} 
Windows : \textless{}tulip\_install\_dir\textgreater{}/bin

\end{itemize}

This path has to be added to the list of Python module search path. To do so, you can add it in the \textbf{PYTHONPATH}
environment variable or add it to the \href{http://docs.python.org/library/sys.html\#sys.path}{\code{sys.path}} list.

Second, your system must be able to find the Tulip C++ libraries in order to use the bindings. These libraries are
also installed in the directory provided above. You have to add this path to the \textbf{LD\_LIBRARY\_PATH} environment variable
on Linux or to the \textbf{PATH} environment variable on Windows.

You should now be able to import the \code{tulip} module through the Python shell. Issue the following command
at the shell prompt to perform that task:

\begin{Verbatim}[commandchars=\\\{\}]
\PYG{g+gp}{\textgreater{}\textgreater{}\textgreater{} }\PYG{k+kn}{from} \PYG{n+nn}{tulip} \PYG{k+kn}{import} \PYG{o}{*}
\end{Verbatim}

Important, if you want to use Tulip algorithms implemented as plugins written in C++ (e.g. graph layout algorithms),
you have to load them before being able to call them (see \code{tlp.applyAlgorithm()}, \code{tlp.Graph.computeLayoutProperty()}, ...).
To load all the Tulip plugins written in C++, you have to execute the following sequence of command:

\begin{Verbatim}[commandchars=\\\{\}]
\PYG{g+gp}{\textgreater{}\textgreater{}\textgreater{} }\PYG{n}{tlp}\PYG{o}{.}\PYG{n}{initTulipLib}\PYG{p}{(}\PYG{p}{)}
\PYG{g+gp}{\textgreater{}\textgreater{}\textgreater{} }\PYG{n}{tlp}\PYG{o}{.}\PYG{n}{loadPlugins}\PYG{p}{(}\PYG{p}{)}
\end{Verbatim}
\phantomsection\label{reference:module-pygraphml}\index{pygraphml (module)}

\section{\texttt{pygraphml} -- API documentation}
\label{reference::doc}\label{reference:pygraphml-api-documentation}\index{Graph (class in Graph)}

\begin{fulllineitems}
\phantomsection\label{reference:Graph.Graph}\pysiglinewithargsret{\strong{class }\code{Graph.}\bfcode{Graph}}{\emph{name='`}}{}
Main class which represent a Graph
\begin{quote}\begin{description}
\item[{Parameters}] \leavevmode\begin{itemize}
\item {} 
\textbf{name} -- name of the graph

\end{itemize}

\end{description}\end{quote}
\index{BFS() (Graph.Graph method)}

\begin{fulllineitems}
\phantomsection\label{reference:Graph.Graph.BFS}\pysiglinewithargsret{\bfcode{BFS}}{\emph{root=None}}{}
Breadth-first search.


\strong{See Also:}


\href{http://en.wikipedia.org/wiki/Breadth-first\_search}{Wikipedia descritpion}


\begin{quote}\begin{description}
\item[{Parameters}] \leavevmode\begin{itemize}
\item {} 
\textbf{root} -- first to start the search

\end{itemize}

\item[{Returns}] \leavevmode
list of nodes

\end{description}\end{quote}

\end{fulllineitems}

\index{DFS\_prefix() (Graph.Graph method)}

\begin{fulllineitems}
\phantomsection\label{reference:Graph.Graph.DFS_prefix}\pysiglinewithargsret{\bfcode{DFS\_prefix}}{\emph{root=None}}{}
Depth-first search.


\strong{See Also:}


\href{http://en.wikipedia.org/wiki/Depth-first\_search}{Wikipedia descritpion}


\begin{quote}\begin{description}
\item[{Parameters}] \leavevmode\begin{itemize}
\item {} 
\textbf{root} -- first to start the search

\end{itemize}

\item[{Returns}] \leavevmode
list of nodes

\end{description}\end{quote}

\end{fulllineitems}

\index{add\_edge() (Graph.Graph method)}

\begin{fulllineitems}
\phantomsection\label{reference:Graph.Graph.add_edge}\pysiglinewithargsret{\bfcode{add\_edge}}{\emph{n1}, \emph{n2}, \emph{directed=False}}{}
\end{fulllineitems}

\index{add\_edge\_by\_label() (Graph.Graph method)}

\begin{fulllineitems}
\phantomsection\label{reference:Graph.Graph.add_edge_by_label}\pysiglinewithargsret{\bfcode{add\_edge\_by\_label}}{\emph{label1}, \emph{label2}}{}
\end{fulllineitems}

\index{add\_node() (Graph.Graph method)}

\begin{fulllineitems}
\phantomsection\label{reference:Graph.Graph.add_node}\pysiglinewithargsret{\bfcode{add\_node}}{\emph{label='`}}{}
\end{fulllineitems}

\index{children() (Graph.Graph method)}

\begin{fulllineitems}
\phantomsection\label{reference:Graph.Graph.children}\pysiglinewithargsret{\bfcode{children}}{\emph{node}}{}
\end{fulllineitems}

\index{edges() (Graph.Graph method)}

\begin{fulllineitems}
\phantomsection\label{reference:Graph.Graph.edges}\pysiglinewithargsret{\bfcode{edges}}{}{}
\end{fulllineitems}

\index{get\_depth() (Graph.Graph method)}

\begin{fulllineitems}
\phantomsection\label{reference:Graph.Graph.get_depth}\pysiglinewithargsret{\bfcode{get\_depth}}{\emph{node}}{}
\end{fulllineitems}

\index{nodes() (Graph.Graph method)}

\begin{fulllineitems}
\phantomsection\label{reference:Graph.Graph.nodes}\pysiglinewithargsret{\bfcode{nodes}}{}{}
\end{fulllineitems}

\index{root() (Graph.Graph method)}

\begin{fulllineitems}
\phantomsection\label{reference:Graph.Graph.root}\pysiglinewithargsret{\bfcode{root}}{}{}
\end{fulllineitems}

\index{set\_root() (Graph.Graph method)}

\begin{fulllineitems}
\phantomsection\label{reference:Graph.Graph.set_root}\pysiglinewithargsret{\bfcode{set\_root}}{\emph{node}}{}
\end{fulllineitems}

\index{set\_root\_by\_attribute() (Graph.Graph method)}

\begin{fulllineitems}
\phantomsection\label{reference:Graph.Graph.set_root_by_attribute}\pysiglinewithargsret{\bfcode{set\_root\_by\_attribute}}{\emph{value}, \emph{attribute='label'}}{}
\end{fulllineitems}

\index{show() (Graph.Graph method)}

\begin{fulllineitems}
\phantomsection\label{reference:Graph.Graph.show}\pysiglinewithargsret{\bfcode{show}}{\emph{show\_label=False}}{}
\end{fulllineitems}


\end{fulllineitems}

\index{Node (class in Node)}

\begin{fulllineitems}
\phantomsection\label{reference:Node.Node}\pysigline{\strong{class }\code{Node.}\bfcode{Node}}{}~\index{children() (Node.Node method)}

\begin{fulllineitems}
\phantomsection\label{reference:Node.Node.children}\pysiglinewithargsret{\bfcode{children}}{}{}
\end{fulllineitems}

\index{edges() (Node.Node method)}

\begin{fulllineitems}
\phantomsection\label{reference:Node.Node.edges}\pysiglinewithargsret{\bfcode{edges}}{}{}
\end{fulllineitems}

\index{parent() (Node.Node method)}

\begin{fulllineitems}
\phantomsection\label{reference:Node.Node.parent}\pysiglinewithargsret{\bfcode{parent}}{}{}
\end{fulllineitems}


\end{fulllineitems}

\index{Edge (class in Edge)}

\begin{fulllineitems}
\phantomsection\label{reference:Edge.Edge}\pysiglinewithargsret{\strong{class }\code{Edge.}\bfcode{Edge}}{\emph{node1}, \emph{node2}, \emph{directed=False}}{}~\index{child() (Edge.Edge method)}

\begin{fulllineitems}
\phantomsection\label{reference:Edge.Edge.child}\pysiglinewithargsret{\bfcode{child}}{}{}
\end{fulllineitems}

\index{directed() (Edge.Edge method)}

\begin{fulllineitems}
\phantomsection\label{reference:Edge.Edge.directed}\pysiglinewithargsret{\bfcode{directed}}{\emph{dir}}{}
\end{fulllineitems}

\index{node() (Edge.Edge method)}

\begin{fulllineitems}
\phantomsection\label{reference:Edge.Edge.node}\pysiglinewithargsret{\bfcode{node}}{\emph{node}}{}
Return the other node

\end{fulllineitems}

\index{parent() (Edge.Edge method)}

\begin{fulllineitems}
\phantomsection\label{reference:Edge.Edge.parent}\pysiglinewithargsret{\bfcode{parent}}{}{}
\end{fulllineitems}

\index{set\_directed() (Edge.Edge method)}

\begin{fulllineitems}
\phantomsection\label{reference:Edge.Edge.set_directed}\pysiglinewithargsret{\bfcode{set\_directed}}{\emph{dir}}{}
\end{fulllineitems}


\end{fulllineitems}

\index{Attribute (class in Attribute)}

\begin{fulllineitems}
\phantomsection\label{reference:Attribute.Attribute}\pysiglinewithargsret{\strong{class }\code{Attribute.}\bfcode{Attribute}}{\emph{name}, \emph{value}, \emph{type='string'}}{}
\end{fulllineitems}

\index{Item (class in Item)}

\begin{fulllineitems}
\phantomsection\label{reference:Item.Item}\pysigline{\strong{class }\code{Item.}\bfcode{Item}}{}~\index{attributes() (Item.Item method)}

\begin{fulllineitems}
\phantomsection\label{reference:Item.Item.attributes}\pysiglinewithargsret{\bfcode{attributes}}{}{}
\end{fulllineitems}


\end{fulllineitems}

\index{Point (class in Point)}

\begin{fulllineitems}
\phantomsection\label{reference:Point.Point}\pysiglinewithargsret{\strong{class }\code{Point.}\bfcode{Point}}{\emph{x=0}, \emph{y=0}, \emph{z=0}}{}~\index{vectorize() (Point.Point method)}

\begin{fulllineitems}
\phantomsection\label{reference:Point.Point.vectorize}\pysiglinewithargsret{\bfcode{vectorize}}{\emph{point}}{}
\end{fulllineitems}


\end{fulllineitems}

\index{GraphMLParser (class in GraphMLParser)}

\begin{fulllineitems}
\phantomsection\label{reference:GraphMLParser.GraphMLParser}\pysigline{\strong{class }\code{GraphMLParser.}\bfcode{GraphMLParser}}{}~\index{parse() (GraphMLParser.GraphMLParser method)}

\begin{fulllineitems}
\phantomsection\label{reference:GraphMLParser.GraphMLParser.parse}\pysiglinewithargsret{\bfcode{parse}}{\emph{fname}}{}
\end{fulllineitems}

\index{write() (GraphMLParser.GraphMLParser method)}

\begin{fulllineitems}
\phantomsection\label{reference:GraphMLParser.GraphMLParser.write}\pysiglinewithargsret{\bfcode{write}}{\emph{graph}, \emph{fname}}{}
\end{fulllineitems}


\end{fulllineitems}



\chapter{Indices and tables}
\label{index:indices-and-tables}\begin{itemize}
\item {} 
\emph{genindex}

\item {} 
\emph{modindex}

\item {} 
\emph{search}

\end{itemize}


\renewcommand{\indexname}{Python Module Index}
\begin{theindex}
\def\bigletter#1{{\Large\sffamily#1}\nopagebreak\vspace{1mm}}
\bigletter{p}
\item {\texttt{pygraphml}}, \pageref{reference:module-pygraphml}
\end{theindex}

\renewcommand{\indexname}{Index}
\printindex
\end{document}
